\documentclass[11pt]{article}

% Language
\usepackage[english]{babel}

% Page size and margins
\usepackage[letterpaper,top=2cm,bottom=2cm,left=3cm,right=3cm,marginparwidth=1.75cm]{geometry}

% Fonts
\usepackage{fontspec} % enables system fonts
\setmainfont{Times New Roman}   % main text in Arial
\setmonofont{Courier New}  % provides Courier font

% Misc packages
\usepackage{microtype}
\usepackage{amsmath}
\usepackage{subfigure}
\usepackage{enumitem}
\usepackage{graphicx}
\usepackage[colorlinks=true, allcolors=blue]

% Bibliography
\usepackage[
  backend=biber,
  style=ieee,
  sorting=none
]{biblatex}
\addbibresource{My first project.bib}
\urlstyle{same}

% Listings
\usepackage{listings}
\usepackage{xcolor}

% Define colors
\definecolor{lightgray}{rgb}{0.95,0.95,0.95}
\definecolor{darkgreen}{rgb}{0,0.5,0}
\definecolor{darkblue}{rgb}{0,0,0.7}
\definecolor{darkred}{rgb}{0.6,0,0}

\lstset{
    language=R,
    basicstyle=\small\ttfamily,   % Courier font
    keywordstyle=\color{blue}\bfseries,
    commentstyle=\color{green!60!black}\itshape,
    backgroundcolor=\color{yellow!15},
    keepspaces=true,
    showstringspaces=false,
    morecomment=[l]{\#},          % <-- tells listings that '#' starts a comment
    numbers=left,
    numberstyle=\tiny,
    stepnumber=1,
    numbersep=5pt,
    frame=single,
    breaklines=true,
    language=bash,
    frameround=tttt
}

\pagestyle{empty}
\pagenumbering{arabic}

\begin{document}

\title{Data Analysis}

The library used for data cleaning
\begin{lstlisting}
# Library 
library(tidyverse)
library(dplyr)
library(haven)
library(tibble)

#Making plot
library(ggplot2)

#imputations
library(mice)

#glmnet and VIF 
library(glmnet)
library(car)

#rexposomes
library(rexposome)
library(Biobase)
set.seed(18)
\end{lstlisting}

\section{Data}
Need to be mindful:

\begin{lstlisting}[language=R]
# Import variables 
exposure_data <- read.csv("/Volumes/Exposome_BIS-I4068/2. Data/10. Clean data/Variable/final_variable_vs5.csv")
# Import out data
out_data <- read.csv("/Users/victoriasjalimcopy/Documents/BIOX7011/Coding/BIOX7011/Final/clean_outcome_m12_vs3.csv", row.names = 1)
\end{lstlisting}

\section{Inspect the data}
We further reduced the variable size due to redundancy or not really important to be included in the analysis. We also combined some variables.

\begin{lstlisting}[language=R]
m#Excluding non-important variables 
exposure_data <- exposure_data %>% 
  select(-c(mpht_t1t2, tothomesibs_t1t2_d, labdurtotmins_d, hresides_t1t2,
            livestock, pctile_bweight_aus_d, numsib,
            loneparent24m_d, starts_with("dist_any"), smkrmhrs,loneparent_t1t2_d,mpmarr_t1t2_d, smkroomdur_12m, mrecdrug12m, precdrug12m))

# Mutate redundant variables 
red_vars <- exposure_data %>% 
  mutate(totpregsmoke = pregsmoket1_t1t2 + pregsmoket2_t1t2) %>%
  mutate(totpregsmoke = case_when(totpregsmoke > 0 ~ 1,
                                  totpregsmoke == 0 ~ 0)) %>%
  mutate(passivesmoke = passive_T2 + passive_T3) %>% 
  mutate(fever = mpfevet1_t1t2 + mpfevet2_t1t2 + mpfevet3) %>% 
  mutate(antibiotic = antibiotic_T1_d + antibiotic_T2_d + antibiotic_T3_d) %>% 
  mutate(decor = ceybed + celivroom + cekitch + cebabyb) %>% 
  mutate(pregalc = rowMeans(cbind(pregalct1_t1t2, pregalct2_t1t2))) 

#Remove the redundant variables
red_vars <- red_vars %>% 
  select(-c(pregsmokerm_t1t2, pregsmoket2_t1t2, passive_T3, passive_T2, pregsmoket1_t1t2, mpfevet3, mpfevet1_t1t2, mpfevet2_t1t2, antibiotic_T3_d, antibiotic_T1_d, antibiotic_T2_d, ceybed, celivroom, cekitch, lsumDEP_DBPs_DEHP_tod_di, pregalct1_t1t2, pregalct2_t1t2, ends_with("byb"))) 

\end{lstlisting}

\subsection{Change the variable names}
Change the variables names so it would be nicer and prettier.

\begin{lstlisting}[language=R]
#Change the colnames - so the variable names is nicer
colnames(red_vars) <- colnames(red_vars) %>% 
  gsub("_di$","",.) %>% 
  gsub("_t1t2$", "", .) %>% 
  gsub("_m12$","12m",.) %>% 
  gsub("hm$", "",.)

#Renaming
name_map <- c("l_8ohdg_w36_adj" = "l8ohdg_w36",
  "l_8ohg_w36_adj" = "l8ohg_w36",
  "ox_stress_gpfs" = "gPFSox",
  "pss_score_301" = "pss_m1",
  "pss_score_312" = "pss_m12",
  "msmkno_12m" = "msmoke_m12", 
  "ssmkno_12m"= "psmoke_m12",
  "pparchilddied" = "parentchildied",
  "hhincome12m_d" = "hhincome12m", 
  "fcobaus_d" = "fcobaus", 
  "mcobaus_d" = "mcobaus", 
  "modebirth_new" = "modebirth", 
  "pc1_modern_healthy" = "healthy_diet",
  "pc2_western_unhealthy" = "unhealthy_diet",
  "pc3_traditional_anglo" = "anglo_diet",
  "pred_pm25_satnowood12m" = "pm25_12m",
  "pred_pm25_satnowood_a" = "pm25_a",
  "mpnonpropancontact" = "animalcontact",
  "dist_maj_rd_m_a" = "roaddist_a", 
  "pred_no212m" = "no2_12m",
  "pred_no2_a" = "no2_a",
  "dist_maj_rd_m12m" = "roaddist_12m",
  "pnt" = "paint",
  "chl" = "chlorine",
  "can" = "candle",
  "sc" = "scents",
  "aero" = "aerosol",
  "airfr" = "airfresh", 
  "hmalcoprob" = "alcoprob",
  "hmdrugprob" = "drugprob",
  "mppetsany" = "anypets",
  "zscore_bweight_aus_d" = "zscore_bweight",
  "maleYN" = "sex")

red_vars <- red_vars %>% 
  rename_with(~ name_map[.x], .cols = names(name_map))

cov <- red_vars %>% 
    select(bisid, sex, zscore_bweight)

red_vars <- red_vars %>% 
  select(-sex, -zscore_bweight)

\end{lstlisting}

\subsection{Modifications}
Modified father employment since everything that <0 usually don't know or refused to answer that is what it better to leave it as NA. 

\begin{lstlisting}
summary(red_vars)

# Modify first 
red_vars$fempl[red_vars$fempl < 0] <- NA
summary(red_vars)
\end{lstlisting}

\subsection{Plot missingness}

\begin{lstlisting}[language=R]
missing_val <- red_vars1 %>% 
  pivot_longer(everything(), names_to = "exposures", values_to = "value") %>% 
  group_by(exposures) %>% 
  summarise(avg = mean(value, na.rm=T),
            sd = sd(value, na.rm = T),
            na = sum(is.na(value)),
            percent_na = (na/697)*100)
            
# Statistics for missing value 
missing_plot <- red_vars1 %>% 
  pivot_longer(everything(), names_to = "exposures", values_to = "value") %>% 
  group_by(exposures) %>% 
  summarise(avg = mean(value, na.rm=T),
            sd = sd(value, na.rm = T),
            na = sum(is.na(value)),
            percent_na = (na/697)*100) %>% 
  mutate(exposures = reorder(exposures, na)) %>% 
  ggplot(aes(x = percent_na, y=exposures, fill = "blue")) +
  geom_col() +
  theme_minimal() +
  theme(
    legend.position = "none",
    axis.text.y  = element_text(size = 7),
    panel.grid.major.x = element_line(colour = "grey85")
  ) +
  labs(x = "Percent NA (%)", y = "Exposures") +
  xlim(0, 100)

missing_plot
\end{lstlisting}


\section{Imputations}
\begin{lstlisting}
#Prepare exposure data excluding the bisid
exp_dat <- red_vars[, -which(names(red_vars) == "bisid")]

#Create the prediction matrix for imputations 
#see tamayo uria et al 
pred_matrix <- quickpred(exp_dat, 
                         mincor = 0.1,
                         minpuc = 0.4)
result <- try({
    mice(exp_dat,
         method = "pmm",
         pred = pred_matrix,
         m = 5,
         seed = 17550,
         printFlag = FALSE)
  }, silent = TRUE)

result_exp <- complete(result,1)

sum(is.na(result_exp)) #0 

# Compare the distributions between two datasets
library(ggforce)
bind_rows(
  red_vars[-1] %>% mutate(.dataset = "A"),
  result_exp %>% mutate(.dataset = "B")
) %>%
  pivot_longer(where(is.numeric), names_to = "var", values_to = "val") %>%
  group_by(.dataset, var) %>%
  ggplot(aes(val, colour = .dataset, fill = .dataset)) +
  geom_density(alpha = 0.25, na.rm = TRUE, adjust = 1) +
  facet_wrap(~ var, nrow = 10, ncol = 10, scales = "free") +
  labs(x = NULL, y = "Density", colour = "Dataset", fill = "Dataset") +
  theme_classic(base_size = 10)
\end{lstlisting}

The analysis are robust as there is not much significant different before and after imputations. 

\subsection{Save the data}
\begin{lstlisting}
# writing the exposures
finalv_imp5 <- cbind(bisid = red_vars$bisid, result_exp)

write.csv(finalv_imp5, "fix_final_vars_148.csv", row.names = F)

# The output
finalout <- out_data %>% 
  rename(sLRTI = cumwhecgh) %>% 
  select(-"cumwhecgh_cat")

# Merge it with covariates
finalout <- merge(finalout, cov, by="bisid")

# Write the outcomes
write.csv(finalout, "out_imp_cov_vs.csv", row.names = F)
 
\end{lstlisting}

\section{Data analysis}
In here we start using rexposome package.

There are 3 requirements for rexposome data:
\begin{itemize}
    \item Exposures: all the exposures every rows representing all the participants, the columns is all the exposures. First columns must be the ID.
    \item Outcome/phenotypes: all the outcomes and its covariates. First columns must be the ID. 
    \item Description: all information about the exposures. The rows is the exposures, make sure you have the columns family to know which family the exposures belongs to. 
\end{itemize}

Bare with me, that there are few trials that I was doing before finalising to the final models. The code would be the same. I decided to use birth weight and sex as covariates from a priori and this covariates would be included in the outcome data. 


\begin{lstlisting}
# Importing all the data 
#Variable
finalvars <- read.csv("/Users/victoriasjalimcopy/Documents/BIOX7011/Coding/BIOX7011/Final/fix_final_vars_148.csv") 

#Description 
desc_new <- read.csv("/Users/victoriasjalimcopy/Documents/BIOX7011/Coding/BIOX7011/Final/description_rename.csv",header = T)

# Outcome 
out_data <- read.csv("/Users/victoriasjalimcopy/Documents/BIOX7011/Coding/BIOX7011/Final/out_imp_cov_vs.csv")
\end{lstlisting}

We tried by creating the exposome first. 
\begin{lstlisting}
eset_imp <- loadExposome(
  exposures = finalvars,
  phenotype = out_data,
  description = desc_new,
  description.famCol = "Family"
  )

eset_imp

#See the missing data
plotMissings(eset_imp, set = "exposures") # For exposures 
\end{lstlisting}

At first, I thought that rexposome package automatically using loadexposome would automatically define the ID. However, the bisid was included in the analysis. After read the manuals, turns out you need to make a path for each files and use the specific parameters to define the ID column (this parameter were not there in the loadexposome). Therefore, we are using the readExposome. 

\begin{lstlisting}
#Writing the path 
final_vars_path <- "/Users/victoriasjalimcopy/Documents/BIOX7011/Coding/BIOX7011/Final/fix_final_vars_148.csv"
final_out_path <- "/Users/victoriasjalimcopy/Documents/BIOX7011/Coding/BIOX7011/Final/out_imp_cov_vs.csv"
final_desc_path <- "/Users/victoriasjalimcopy/Documents/BIOX7011/Coding/BIOX7011/Final/description_rename.csv"

#using the rexposome
eset_final <- readExposome(exposures = final_vars_path,
                           phenotype = final_out_path, 
                           description = final_desc_path,
                           description.famCol = "Family", 
                           description.expCol = "Name",
                           exposures.samCol = "bisid", # Define the col for bisid
                           phenotype.samCol = "bisid")  # Define the col for bisid
eset_final
# 85 exposures 
plotFamily(eset_final, family = "all") 
\end{lstlisting}

\subsection{Transformations}

At first, I want to use box-cox transformations. However, it kept showing me an error and this is because we do have some negative value such as for PC1 for diets and phtalates. After few research, that is why we are doing yeo-johnson transformations, the extension of box-cox that can handle negative values.

\begin{lstlisting}
library(e1071)
library(car)

# Function for the transformation  
expo_best_trans <- function(var) {
  
  transformed_df <- var
  
  for (x in colnames(var)) {
    if (is.numeric(var[[x]])) {
      yj <- powerTransform(var[[x]], family = "yjPower")
      transformed_df[[x]] <- yjPower(var[[x]], lambda = yj$lambda)
    }
  }
  return(transformed_df)
}

# Transform the continuous data 
eset_imp_trans <- trans(eset_final, fun = expo_best_trans)
eset_imp_trans

#Standardize 
eset_imp_t_s <- standardize(eset_imp_trans, method = "normal")
eset_imp_t_s
plotFamily(eset_imp_t_s, family = "all")
\end{lstlisting}

\subsection{Correlations}

\begin{lstlisting}
#Correlation
eset_imp_t_s_c <- correlation(eset_imp_t_s, use = "pairwise.complete.obs", method.cor = "pearson")

#Plot
plotCorrelation(eset_imp_t_s_c, type = "matrix")

\end{lstlisting}

Correlations has other visualizations, which is the circular diagram. But the as the variables is really big, it didn't work. The matrix didn't really give meaningful relationship Needs to find way to do dimensionality reductions.

\subsection{PCA}

\begin{lstlisting}
#Trying with all T
pca1 <- pca(eset_imp_t_s, pca = T)
plotPCA(pca1, set = "all")

plotPCA(pca1, set = "exposures", show.exposures = TRUE)

#Plot the exposures
plotEXP(pca1)

#Plot the phenotype
plotPHE(pca1)
\end{lstlisting}

The first PCA (pca1) has 65 principal components. Visualization methods implemented in rexposome only allows to visualize the first 10 principal components.

\subsection{ExWAS}

\begin{lstlisting}
#Look at the phenotype 
phenotypeNames(eset_imp_t_s)

#for the cumwheeze
ex_cum <- exwas(eset_imp_t_s, formula = cumwhecgh ~ sex + zscore_bweight, family = "gaussian")
ex_cum

plotExwas(ex_cum)
plotEffect(ex_cum)
\end{lstlisting}

Just realized that rexposome excluded any non-continuous data in the analysis. I am scared we might miss anything since most of our variables are categorical. Need to find a way to do this.

\section{New Eset}

\begin{lstlisting}
eset_final <- readExposome(exposures = final_vars_path,
                           phenotype = final_out_path, 
                           description = final_desc_path,
                           description.famCol = "Family", 
                           description.expCol = "Name",
                           exposures.samCol = "bisid",
                           phenotype.samCol = "bisid", 
                           exposures.asFactor = 1) # Included all data that at least have one factor
eset_final
plotFamily(eset_final, family = "all") 
\end{lstlisting}


\subsection{Pre-processing}
Same pre-processing done from before. 

\begin{lstlisting}
# Transform the continuous data 
eset_imp_trans <- trans(eset_final, fun = expo_best_trans)
eset_imp_trans

#Standardize 
eset_imp_t_s <- standardize(eset_imp_trans, method = "normal")
eset_imp_t_s
plotFamily(eset_imp_t_s, family = "all")
\end{lstlisting}

\subsection{ExWAS}
\begin{lstlisting}
#Look at the phenotype 
phenotypeNames(eset_imp_t_s)

#for the cumwheeze
ex_cum <- exwas(eset_imp_t_s, formula = sLRTI ~ sex + zscore_bweight, family = "gaussian")
ex_cum

plotExwas(ex_cum)
plotEffect(ex_cum)
\end{lstlisting}

\subsection{InExWAS}
\begin{lstlisting}[language=R]
inv_ex <- invExWAS(eset_imp_t_s, formula = ~sex + zscore_bweight)
#head(extract(inv_ex))

# The visualisation 
clr <- rainbow(length(familyNames(eset_imp_t_s)))
names(clr) <- familyNames(eset_imp_t_s)
plotExwas(inv_ex, color = clr)
\end{lstlisting}

The curse of dimensionality happened in here. In rexposome, they used the method tef or threshold effective test, where we create our own p-value depended on the size of the data. Increase number of exposure will lead to smaller p-value. When testing many univariate tests (e.g., testing hundreds of exposures for association with one health outcome), the chance of getting false positives increases. Therefore, we didn't find a significant result despite probably some of the variables actually have a low p-value, because didn't passed the tef (The TEF was 7.74e-04)

That is why we further separate the analysis into postnatal and prenatal. 

\section{Dictionary}
To have more clarity on our graph, however, there should be no spaces in the variables name, we will make dictionary that corresponds to the variables name. 

\begin{lstlisting}
## Dictionary 

dict_name <- c(
  pregalc = "Prenatal Alcohol Intake",
  no2_12m = "NO2 (12m)",
  no2_a = "NO2 (pregnancy)",
  pm25_12m = "PM2.5 (12m)",
  pm25_a = "PM2.5 (pregnancy)",
  roaddist_12m = "Road distance (12m)",
  roaddist_a = "Road distance (pregnancy)",
  atreg12m = "aTregs (12m)",
  GlycA_a = "GlycA (pregnancy)",
  GlycA_b = "GlycA (birth)",
  GlycA12m = "GlycA (12m)",
  gPFSox = "gPFSox",
  l8ohdg_w36 = "8-OH-dG",
  l8ohg_w36 = "8-OH-Gua",
  rtreg12m = "rTregs (12m)",
  bfany = "Any Breastfed",
  healthy_diet = "Healthy Diet",
  unhealthy_diet = "Unhealthy Diet",
  anglo_diet = "Anglo Diet",
  alcoprob = "Alcohol problem",
  drugprob = "Drugs problem",
  eds_score = "EDS Score",
  financialprob = "Financial problem",
  malcot12m = "Mother alcohol intake (12m)",
  palcot12m = "Father alcohol intake (12m)",
  parentchildied = "Any parents/child died",
  policecourt = "Police/legal problem",
  pss_m1 = "PSS (1m)",
  pss_m12 = "PSS (12m)",
  reldied = "Relative died",
  stolenval = "Robbery",
  animalcontact = "Animal contact (not house)",
  antibiotic = "Pregnancy antibiotic used",
  anypets = "Pets",
  disinfhmt1 = "Disinfectant (pregnancy)",
  fever = "Fever (pregnancy)",
  hresidesadult = "Number of Adults",
  hresideschild = "Number of children",
  modebirth = "Mode of birth",
  mpfarm = "Farm work",
  buf_500 = "Green space",
  NDVI = "NDVI",
  aerosol = "Aerosol",
  airfresh = "Air fresheners",
  candle = "Candle",
  carpcl = "Carpet cleaner",
  chlorine = "Chlorine",
  decor = "Decoration in the house",
  disinf = "Disinfectant",
  fabsofbabycl = "Fabric softener",
  fumany = "Any fumigations", 
  glasscl = "Glass cleaner",
  laubabycl = "Liquid laundry",
  mcoil = "Mosquito coil",
  paint = "Paint",
  pest = "Pest",
  repel = "Repellent",
  scents = "Scented product",
  stainbabycl = "Stain remover",
  turp = "Turpentine",
  shannon_a = "Shannon index (pregnancy)",
  shannon12m = "Shannon index (12m)",
  fcobaus = "Father born in Australia",
  feducation = "Father education",
  fempl = "Father employment",
  hhincome12m = "Household income",
  mage = "Mother Age",
  mcobaus = "Mother born in Australia",
  meducation = "Mother education",
  mempl = "Mother employment",
  seifa_disad = "SEIFA disadvantages score",
  msmoke_m12 = "Mother smoke (12m)",
  passivesmoke = "Passive smoke (pregnancy)",
  psmoke_m12 = "Father smoke (12m)",
  totpregsmoke = "Mother smoke (pregnancy)",
  zscore_bweight = "Birthweight",
  cumwhecgh = "sLRTI"
)

pretty_label <- function(x) {
  ifelse(x %in% names(dict_name), dict_name[x], x)}

# I ask chat gpt for more colors because IDK what other colors R have
exposure_colors <- c(
  "Family adversity" = "#FF4500",      # Orange red
  "Biomarker" = "#B8860B",            # Crimson
  "Diet" = "#8B4513",                 # Saddle brown
  "SEP" = "#000080",                  # Navy blue
  "Smoking" = "hotpink",              # Indigo
  "Ambient air quality" = "#800080",   # Purple
  "MGM" = "orange",                  # Deep pink
  "Indoor air quality " = "darkgrey",
  "Green space" = "green",           # Dark green
  "FEME" = "#008B8B",                 # Dark cyan
  "Phenols" = "darkgreen",              # Dark olive green
  "Alcohol intake" = "#DC143C",        # Dark goldenrod
  "Phtalates " = "red"            
)
\end{lstlisting}

\section{Postnatal}
\begin{lstlisting}
#Updated the description
desc_new <- read.csv("description_rename.csv")

#Select the postnatal variable 
vars_post <- desc_new %>% 
  filter(Status == "postnatal") %>% 
  pull(Name)
vars_post <- c("bisid", vars_post)

#Write the variables documents 
post_vars <- finalvars[ ,vars_post]
post_vars[] <- lapply(post_vars, as.numeric)
write.csv(post_vars, "exposome_post_vars_vs.csv", row.names = F)

#Description for postnatal
vars_post <- vars_post[vars_post != "bisid"]
desc_post <- desc_new[desc_new$Name %in% vars_post,]
write.csv(desc_post, "description_update_post.csv")

final_vars_post_path <- "/Users/victoriasjalimcopy/Documents/BIOX7011/Coding/BIOX7011/Final/exposome_post_vars_vs.csv"

final_desc_post_path <- "/Users/victoriasjalimcopy/Documents/BIOX7011/Coding/BIOX7011/Final/description_update_post.csv"

#using the rexposome
eset_post <- readExposome(
  exposures = final_vars_post_path,
  phenotype = final_out_path, 
  description = final_desc_post_path,
  description.famCol = "Family", 
  description.expCol = "Name",
  exposures.samCol = 1,
  phenotype.samCol = 1,
  exposures.asFactor = 1)
eset_post

# Transform the continuous data 
eset_post_t <- trans(eset_post, fun = expo_best_trans)
eset_post_t

#Standardize 
eset_post_ts <- standardize(eset_post_t, method = "normal")
eset_post_ts

# Summary Statistics 
plotFamily(eset_post, family = "all")
plotFamily(eset_post_ts, family = "all")
\end{lstlisting}

\subsection{Correlations}
We are doing Pearson's correlations. If we wanted to changed the variable names, we need to make sure to changed multiple data frame, since mismatch of names will not produced a graph. To be mindful: assay data (correlation matrix) and feature data (contains family),  

\begin{lstlisting}
#Correlation
eset_post_c <- correlation(eset_post_ts)
eset_post_c

# Changing the row names so it will have pretty names 
# Change the rownames for the feature data first 
rownames(fData(eset_post_c)) <- dict_name[rownames(fData(eset_post_c))]
# Change the rownames for the the assay data
rownames(eset_post_c@assayData$corr) <- dict_name[rownames(eset_post_c@assayData$corr)]
# Also the colnames 
colnames(eset_post_c@assayData$corr) <- dict_name[colnames(eset_post_c@assayData$corr)]

#Plot
plotCorrelation(eset_post_c, type = "circos")
\end{lstlisting}

\subsection{PCA}
\begin{lstlisting}
#Trying with all T
pca_post <- pca(eset_post_ts, pca = T)
plotPCA(pca_post, set = "all")

plotPCA(pca_post, set = "exposures", show.exposures = FALSE) + 
  theme(legend.position = "right") +
  theme_minimal()
  
#Plot the exposures
plotEXP(pca_post) + 
    # Change the scale
  ggplot2::scale_fill_gradient2(low= "steelblue", mid = "white", high = "red") + 
  # Change the variables name
  ggplot2::scale_y_discrete(labels = pretty_label) +
  labs(x = "PCs", y = "Exposures")
  
#Plot the phenotype
plotPHE(pca_post) +
  labs(x = "PCs", y = "Phenotype")
\end{lstlisting}

\subsection{ExWAS}
Actually I run it without changing the name first, then I included the rename all the variables again. 
\begin{lstlisting}
#for the cumwheeze
ex_cum_post <- exwas(eset_post_ts, formula = sLRTI ~ sex + zscore_bweight, family = "gaussian", warnings = F)
ex_cum_post
head(ex_cum_post@comparison)
#plotExwas(ex_cum_post) +
 # geom_point(size=3) + scale_color_manual(values = exposure_colors)
#plotEffect(ex_cum_post) 

exwas_post <- as.data.frame(ex_cum_post@comparison) %>% 
  mutate(label = dplyr::recode(rownames(ex_cum_post@comparison),
                        GlycA12m = "GlycA (12m)",
                        financialprob = "Financial Problem",
                        fumany = "Any fumigations",
                        msmoke_m12 = "Mother smoke (12m)"), 
         sig = pvalue < 0.05)

# Volcano plot
#plotVolcano(ex_cum_pre) +
ggplot(exwas_post, aes(x=as.numeric(effect), y= -log10(as.numeric(pvalue)),
                                   colour = sig)) +
  geom_point() +
  labs(x = "Effect", y = expression(-log[10](p-value))) +
  # Create the threshold p-value
  geom_hline(yintercept = -log10(1.46e-03), 
             col="darkred", linetype = "dashed") +
  geom_hline(yintercept = -log10(0.05), 
             col="darkgreen", linetype = "dashed") +
  geom_vline(xintercept = 0, col="blue", linetype = "dashed") +
  scale_colour_manual(values = c(`FALSE` = "black", 
                                 `TRUE` = "red")) +
  coord_cartesian(xlim = c(-5, 5)) +
  guides(color = "none", shape = "none") +
  theme_minimal() +
  theme(panel.border = element_rect(colour = "black", 
                                    linewidth = 0.8, fill = NA),
        panel.background = element_rect(fill = "white", 
                                        color = NA),
        plot.background  = element_rect(fill = "white", 
                                        color = NA),
        legend.position = "none") +
  ggrepel::geom_text_repel(data = subset(exwas_post, sig), 
                           aes(label = label),
                           size = 4,
                           box.padding = unit(0.35, "lines"),
                           point.padding = unit(0.35, "lines"),
                           max.overlaps  = Inf,
                           segment.color = "black") +
  coord_cartesian(xlim = c(-0.5, 0.5)) +
  labs(x="Effect")

#write.csv(ex_cum_post@comparison, "Result_postnatal_exwas.csv")
\end{lstlisting}

\subsection{InExWAS}
\begin{lstlisting}
inexwas_cum <- invExWAS(eset_post_ts, formula = ~zscore_bweight + sex)
#head(extract(inexwas_cum))
inexwas_cum

# The visualisation
# This is the first visualisations but I don't like it
plotExwas(inexwas_cum) + 
  geom_point(size=3) + scale_color_manual(values = exposure_colors) 

#write.csv(inexwas_cum@comparison, "Result_post_inexwas.csv")

# Better visualisations
inexwas_post <- as.data.frame(inexwas_cum@comparison) %>% 
  mutate(label = dplyr::recode(rownames(inexwas_cum@comparison),
                               pss_m12 = "PSS (12m)",
                               financialprob = "Financial Problem",
                               stolenval = "Robbery",
                               fumany = "Any fumigations",
                               msmoke_m12 = "Mother smoke (12m)",
                               buf_500 = "Green space",
                               pm25_12m = "PM2.5",
                               drugprob = "Drugs problem",
                               repel = "Repellent"), 
         sig = pvalue < 0.05)

# Volcano plot
#plotVolcano(ex_cum_pre) +
ggplot(inexwas_post, aes(x=as.numeric(effect), y= -log10(as.numeric(pvalue)),
                                   colour = sig)) +
  geom_point() +
  labs(x = "Effect", y = expression(-log[10](p-value))) +
  # Create the threshold p-value
  geom_hline(yintercept = -log10(1.46e-03), 
             col="darkred", linetype = "dashed") +
  geom_hline(yintercept = -log10(0.05), 
             col="darkgreen", linetype = "dashed") +
  geom_vline(xintercept = 0, col="blue", linetype = "dashed") +
  scale_colour_manual(values = c(`FALSE` = "black", 
                                 `TRUE` = "red")) +
  coord_cartesian(xlim = c(-0.25, 0.25)) +
  guides(color = "none", shape = "none") +
  theme_minimal() +
  theme(panel.border = element_rect(colour = "black", 
                                    linewidth = 0.8, fill = NA),
        panel.background = element_rect(fill = "white", 
                                        color = NA),
        plot.background  = element_rect(fill = "white", 
                                        color = NA),
        legend.position = "none") +
  ggrepel::geom_text_repel(data = subset(inexwas_post, sig), 
                           aes(label = label),
                           size = 4,
                           box.padding = unit(0.35, "lines"),
                           point.padding = unit(0.35, "lines"),
                           max.overlaps  = Inf,
                           segment.color = "black",
                           min.segment.length = 0) +
  coord_cartesian(xlim = c(-0.5, 0.5)) +
  labs(x="Effect")

#ggsave("Result/inexwas_volcano.png", v_inexwas_po)
\end{lstlisting}

\subsection{MexWAS}
We are using elastic net. Using lambda min and alpha 0.5, with 10 CV.
\begin{itemize}
    \item If coefficient > 0: higher values of that variable are associated with an increased value of the outcome
    \item  If coefficient < 0: higher values of the predictor are associated with a decrease in the outcome.
\end{itemize}
\begin{lstlisting}
# Get all the Exposures 
post_vars_st <- expos(eset_post_ts)  
# Obtain the outcome for postnatal 
Y <- pData(eset_post_ts)
Y <- Y$cumwhecgh                           

# Do cross validations (to find the lambda)
cv_enet_post <- cv.glmnet(as.matrix(post_vars_st), Y, family = "gaussian", alpha = 0.5)
# Enet with selected lambda 
enet_post <- glmnet(post_vars_st, Y, alpha = 0.5, lambda = cv_enet_post$lambda.min)

# See which variables are important
coef(enet_post, s = cv_enet_post$lambda.min)

# Visualisations
coef(enet_post, s = cv_enet_post$lambda.min) %>% 
  as.matrix() %>% as.data.frame() %>%
  setNames("coef") %>% 
  rownames_to_column("feature") %>%
  filter(feature != "(Intercept)") %>% 
  ggplot(aes(x = feature, y = coef, fill = coef)) +
  geom_col() +
  scale_x_discrete(labels = pretty_label) +   
  scale_fill_gradient2(low = "steelblue", mid = "white", high = "red", midpoint = 0) +
  coord_flip() +
  labs(x = NULL, y = "Coefficient", fill = "Value") +
  theme_minimal(base_size = 12)

\end{lstlisting}

\section{Prenatal}
We are begin to analyze about prenatal exposures. 

\begin{lstlisting}
#Select the postnatal variable 
vars_pre <- desc_new %>% 
  filter(Status == "prenatal") %>% 
  pull(Name)
vars_pre <- c("bisid", vars_pre)

#Write the variables documents 
pre_vars <- finalvars[ ,vars_pre]
pre_vars[] <- lapply(pre_vars, as.numeric)
write.csv(pre_vars, "exposome_pre_vars_vs.csv", row.names = F)

#Description for prenatal
vars_pre <- vars_pre[vars_pre != "bisid"]
desc_pre <- desc_new[desc_new$Name %in% vars_pre,]
write.csv(desc_pre, "description_update_pre.csv")

final_vars_pre_path <- "/Users/victoriasjalimcopy/Documents/BIOX7011/Coding/BIOX7011/Final/exposome_pre_vars_vs.csv"

final_desc_pre_path <- "/Users/victoriasjalimcopy/Documents/BIOX7011/Coding/BIOX7011/Final/description_update_pre.csv"

#using the rexposome
eset_pre <- readExposome(
  exposures = final_vars_pre_path,
  phenotype = final_out_path, 
  description = final_desc_pre_path,
  description.famCol = "Family", 
  description.expCol = "Name",
  exposures.samCol = 1,
  phenotype.samCol = 1,
  exposures.asFactor = 1)
eset_pre

# Transform the continuous data 
eset_pre_t <- trans(eset_pre, fun = expo_best_trans)
eset_pre_t

#Standardize 
eset_pre_ts <- standardize(eset_pre_t, method = "normal")
eset_pre_ts
\end{lstlisting}

\subsection{Correlations}
Same as above in 7.1. However, in here, there is some variables (i.e., like the phenols ), that wasn't in the dictionary, which cause an error. We need to do different way.  

\begin{lstlisting}
#Correlation
eset_pre_c <- correlation(eset_pre_ts)
eset_pre_c

# Only change the name if it's contain in the dictionary
if (!is.null(dict_name) && length(dict_name) > 0){
  # For fData rownames
  new_fdata_names <- rownames(fData(eset_pre_c))
  idx <- new_fdata_names %in% names(dict_name)
  new_fdata_names[idx] <- dict_name[new_fdata_names[idx]]
  rownames(fData(eset_pre_c)) <- new_fdata_names
  
  # For rownames in correlation matrix 
  new_row_names <- rownames(eset_pre_c@assayData$corr)
  idx <- new_row_names %in% names(dict_name)
  new_row_names[idx] <- dict_name[new_row_names[idx]]
  rownames(eset_pre_c@assayData$corr) <- new_row_names
  
  # For colnames in correlations matrix 
  new_col_names <- colnames(eset_pre_c@assayData$corr)
  idx <- new_col_names %in% names(dict_name)
  new_col_names[idx] <- dict_name[new_col_names[idx]]
  colnames(eset_pre_c@assayData$corr) <- new_col_names
}

#Plot
plotCorrelation(eset_pre_c, type = "circos")
\end{lstlisting}

\subsection{PCA}
\begin{lstlisting}
#Trying with all T
pca_pre <- pca(eset_pre_ts, pca = T)
plotPCA(pca_pre, set = "all")

# Biplot
plotPCA(pca_pre, set = "exposures", show.exposures = FALSE) +
  theme(legend.position = "right") +
  theme_minimal()

#Plot the exposures
plotEXP(pca_pre) + 
  ggplot2::scale_fill_gradient2(low= "steelblue", 
                                mid = "white", high = "red") +
  scale_y_discrete(labels = pretty_label) +
  labs(y = "Exposures", x="PCs") +
  geom_point(size = 5) +  # increase bullet (point) size
  theme(
    text = element_text(size = 18),  # overall text size
    axis.title = element_text(size = 18),  # axis labels
    axis.text = element_text(size = 14),   # axis tick labels
    legend.title = element_text(size = 16), # legend title
    legend.text = element_text(size = 14)   # legend labels
  ) +
  guides(colour = guide_legend(override.aes = list(size = 6)))

#Plot the phenotype
plotPHE(pca_pre) + labs(y = "Phenotype", x="PCs") 
\end{lstlisting}

\subsection{ExWAS}
\begin{lstlisting}
ex_cum_pre <- exwas(eset_pre_ts, formula = sLRTI ~ sex + zscore_bweight, family = "gaussian", warnings = F)
ex_cum_pre
plotExwas(ex_cum_pre) + ggplot2::geom_point(size = 3) + scale_color_manual(values = exposure_colors)
plotEffect(ex_cum_pre)

# Write the result 
ex_pre_res <- as.data.frame(ex_cum_pre@comparison)
rownames(ex_pre_res) <- pretty_label(rownames(ex_pre_res)) 
ex_pre_res <- rownames_to_column(ex_pre_res, var="variable")
ex_pre_res <- merge(ex_pre_res, fam, by="variable")
write.csv(ex_pre_res, "Exwas_pre_res.csv")

# Visualisations
# Mark the variables as significant
exwas_pre <- as.data.frame(ex_cum_pre@comparison) %>% 
  mutate(label = dplyr::recode(rownames(ex_cum_pre@comparison),
                        hresideschild = "Number of children",
                        unhealthy_diet = "Unhealthy Diet",
                        BPA = "BPA",
                        seifa_disad = "SEIFA disadvantages",
                        GlycA_a = "GlycA (pregnancy)",
                        pregalc = "Prenatal Alcohol Intake",
                        healthy_diet = "Healthy Diet"), 
         sig = pvalue < 0.05)

# Volcano plot
#plotVolcano(ex_cum_pre) +
ggplot(exwas_pre, aes(x=as.numeric(effect), y= -log10(as.numeric(pvalue)),
                                   colour = sig)) +
  geom_point() +
  labs(x = "Effect", y = expression(-log[10](italic(P)))) +
  # Create the threshold p-value
  geom_hline(yintercept = -log10(1.46e-03), 
             col="darkred", linetype = "dashed") +
  geom_hline(yintercept = -log10(0.05), 
             col="darkgreen", linetype = "dashed") +
  geom_vline(xintercept = 0, col="blue", linetype = "dashed") +
  scale_colour_manual(values = c(`FALSE` = "black", 
                                 `TRUE` = "red")) +
  coord_cartesian(xlim = c(-5, 5)) +
  guides(color = "none", shape = "none") +
  theme_minimal() +
  theme(panel.border = element_rect(colour = "black", 
                                    linewidth = 0.8, fill = NA),
        panel.background = element_rect(fill = "white", 
                                        color = NA),
        plot.background  = element_rect(fill = "white", 
                                        color = NA),
        legend.position = "none") +
  ggrepel::geom_text_repel(data = subset(exwas_pre, sig), 
                           aes(label = label),
                           size = 4,
                           box.padding = unit(0.35, "lines"),
                           point.padding = unit(0.35, "lines"),
                           max.overlaps  = Inf,
                           segment.color = "black") +
  coord_cartesian(xlim = c(-0.5, 0.5)) +
  labs(x="Effect")
\end{lstlisting}

\subsection{InExWAS}
\begin{lstlisting}
inexwas_cum_pre <- invExWAS(eset_pre_ts, formula = ~zscore_bweight + sex)
#head(extract(inexwas_cum_pre))

# Write the result
inex_pre_res <- as.data.frame(inexwas_cum_pre@comparison)
rownames(inex_pre_res) <- pretty_label(rownames(inex_pre_res)) 
inex_pre_res <- rownames_to_column(inex_pre_res, var="variable")
inex_pre_res <- merge(inex_pre_res, fam, by="variable")
write.csv(inex_pre_res, "inExwas_pre_res.csv")

# The visualisation 
plotExwas(inexwas_cum_pre) + geom_point(size=3) + scale_color_manual(values = exposure_colors)

# Create volcano plot
inexwas_pre <- as.data.frame(inexwas_cum_pre@comparison) %>% 
  mutate(label = dplyr::recode(rownames(inexwas_cum_pre@comparison),
                        hresideschild = "Number of children",
                        unhealthy_diet = "Unhealthy Diet",
                        BPA = "BPA",
                        seifa_disad = "SEIFA disadvantages",
                        GlycA_a = "GlycA (pregnancy)"), 
         sig = pvalue < 0.05)

ggplot(inexwas_pre, aes(x=effect, y=-log10(pvalue),
                                   colour = sig)) +
  # Create the threshold p-value
  geom_hline(yintercept = -log10(1.46e-03), 
             col="darkred", linetype = "dashed") +
  geom_hline(yintercept = -log10(0.05), 
             col="darkgreen", linetype = "dashed") +
  geom_vline(xintercept = 0, col="blue", linetype = "dashed") + 
  geom_point() +
  # Create the label
  labs(x = "Effect", y = expression(-log[10](P-value))) +
  scale_colour_manual(values = c(`FALSE` = "black", 
                                 `TRUE` = "red")) +
  coord_cartesian(xlim = c(-0.3, 0.3)) +
  guides(color = "none", shape = "none") + 
   ggrepel::geom_text_repel(data = subset(inexwas_pre, sig), 
                           aes(label = label),
                           size = 4,
                           box.padding = unit(0.5, "lines"),
                           point.padding = unit(0.35,"lines"),
                           max.overlaps  = Inf,
                           segment.color = "black",
                           min.segment.length = 0) +
  theme_minimal() +
  theme(panel.border = element_rect(colour = "black", 
                                    linewidth = 0.8, 
                                    fill = NA),
        panel.background = element_rect(fill = "white", 
                                        color = NA),
        plot.background  = element_rect(fill = "white", 
                                        color = NA))

# Write the result 
write.csv(inexwas_cum_pre@comparison, "Result_inexwas_prenatal.csv") 
\end{lstlisting}

\subsection{MExWAS}
Understanding elastic net:
\begin{itemize}
    \item If coefficient > 0: higher values of that variable are associated with an increased value of the outcome
    \item  If coefficient < 0: higher values of the predictor are associated with a decrease in the outcome.
\end{itemize}

\begin{lstlisting}
#Get the data
pre_vars_st <- expos(eset_pre_ts)

#glmnet
cv_pre <- cv.glmnet(as.matrix(pre_vars_st), Y, 
                    family = "gaussian", alpha = 0.5)
enet_pre_cum <- glmnet(pre_vars_st, Y, alpha = 0.5, lambda = cv_pre$lambda.min)

# Plot the graph 
coef(enet_pre_cum, s = cv_pre$lambda.min) %>% 
  as.matrix() %>% as.data.frame() %>%
  setNames("coef") %>% 
  rownames_to_column("feature") %>%
  filter(feature != "(Intercept)") %>% 
  ggplot(aes(x=feature, y=coef, fill=coef)) +
  scale_x_discrete(labels=pretty_label) +
  geom_col() +
  scale_fill_gradient2(low= "steelblue", mid = "white", 
                       high = "red") + 
  coord_flip() +
  labs(
    x = NULL, y = "Coefficient",
    fill = "Value"
  ) +
  theme_minimal(base_size = 12)
\end{lstlisting}

\section{Sensitivity Analyses}
The biomarkers and MGM can be a confounding factors between the exposures and the outcome, we want to know if the analysis is robust or not.

\begin{lstlisting}
# Update the variables 
desc_noox <- desc_new %>% 
  filter(!Family %in% c("Biomarker", "MGM")) %>% 
  filter(!(Name %in% c("mage")))

\end{lstlisting}

\subsection{Postnatal}
\subsubsection{Pre-processing & ExwAS}
\begin{lstlisting}
#Select the postnatal variable 
vars_post_cov <- desc_noox %>% 
  filter(Status == "postnatal") %>% 
  pull(Name)
vars_post_cov <- c("bisid", vars_post_cov)

#Write the variables documents 
post_vars_cov <- finalvars[ ,vars_post_cov]
post_vars_cov[] <- lapply(post_vars_cov, as.numeric)
write.csv(post_vars_cov, "exposome_post_vars_cov_vs.csv", row.names = F)

#Description for postnatal
vars_post_cov <- vars_post_cov[vars_post_cov != "bisid"]
desc_post_cov <- desc_noox[desc_noox$Name %in% vars_post_cov,]
write.csv(desc_post_cov, "description_update_post_cov.csv")

final_vars_post_cov_path <- "/Users/victoriasjalimcopy/Documents/BIOX7011/Coding/BIOX7011/Final/exposome_post_vars_cov_vs.csv"

final_desc_post_cov_path <- "/Users/victoriasjalimcopy/Documents/BIOX7011/Coding/BIOX7011/Final/description_update_post_cov.csv"

#using the rexposome
eset_post_cov <- readExposome(
  exposures = final_vars_post_cov_path,
  phenotype = final_out_cov_path, 
  description = final_desc_post_cov_path,
  description.famCol = "Family", 
  description.expCol = "Name",
  exposures.samCol = "bisid",
  phenotype.samCol = "bisid",
  exposures.asFactor = FALSE)
eset_post_cov

# Transform the continuous data 
eset_post_t_cov <- trans(eset_post_cov, fun = expo_best_trans)
#Standardize 
eset_post_ts_cov <- standardize(eset_post_t_cov, method = "normal")
eset_post_ts

# Summary Statistics 
plotFamily(eset_post_ts_cov, family = "all")

# EWAS 
ex_cum_pos_cov <- exwas(eset_post_ts_cov, formula = cumwhecgh ~ sex + zscore_bweight, family = "gaussian", warnings = F)
ex_cum_pos_cov
plotExwas(ex_cum_pos_cov)
plotEffect(ex_cum_pos_cov)

exwas_pos_cov <- as.data.frame(ex_cum_pos_cov@comparison) %>% 
   mutate(label = dplyr::recode(rownames(ex_cum_pos_cov@comparison),
                        GlycA12m = "GlycA (12m)",
                        financialprob = "Financial Problem",
                        fumany = "Any fumigations",
                        msmoke_m12 = "Mother smoke (12m)"), 
         sig = pvalue < 0.05)

# Volcano plot
#plotVolcano(ex_cum_pre) +
ex_post_sens <- ggplot(exwas_pos_cov, aes(x=as.numeric(effect), y= -log10(as.numeric(pvalue)),
                                   colour = sig)) +
  geom_point() +
  labs(x = "Effect", y = expression(-log[10](italic(P)))) +
  # Create the threshold p-value
  geom_hline(yintercept = -log10(1.46e-03), 
             col="darkred", linetype = "dashed") +
  geom_hline(yintercept = -log10(0.05), 
             col="darkgreen", linetype = "dashed") +
  geom_vline(xintercept = 0, col="blue", linetype = "dashed") +
  scale_colour_manual(values = c(`FALSE` = "black", 
                                 `TRUE` = "red")) +
  coord_cartesian(xlim = c(-5, 5)) +
  guides(color = "none", shape = "none") +
  theme_minimal() +
  theme(panel.border = element_rect(colour = "black", 
                                    linewidth = 0.8, fill = NA),
        panel.background = element_rect(fill = "white", 
                                        color = NA),
        plot.background  = element_rect(fill = "white", 
                                        color = NA),
        legend.position = "none") +
  ggrepel::geom_text_repel(data = subset(exwas_pos_cov, sig), 
                           aes(label = label),
                           size = 4,
                           box.padding = unit(0.35, "lines"),
                           point.padding = unit(0.35, "lines"),
                           max.overlaps  = Inf,
                           segment.color = "black") +
  coord_cartesian(xlim = c(-0.5, 0.5)) +
  labs(x="Effect")


ggsave("post_exwas_sens.png", ex_post_sens, width = 8, height = 6, dpi = 300)

\end{lstlisting}

\subsubsection{InExWAS}
\begin{lstlisting}
# Inexwas
inexwas_cum_post_c <- invExWAS(eset_post_ts_cov, formula = ~zscore_bweight + sex)

#head(extract(inexwas_cum_pre))
inexwas_pos_cov <- as.data.frame(inexwas_cum_post_c@comparison) %>% 
   mutate(label = dplyr::recode(rownames(inexwas_cum_post_c@comparison),
                        pss_m12 = "PSS (12m)",
                               financialprob = "Financial Problem",
                               stolenval = "Robbery",
                               fumany = "Any fumigations",
                               msmoke_m12 = "Mother smoke (12m)",
                               buf_500 = "Green space",
                               pm25_12m = "PM2.5",
                               drugprob = "Drugs problem",
                               repel = "Repellent"), 
         sig = pvalue < 0.05)

# Volcano plot
post_inexwas_sens <- ggplot(inexwas_pos_cov, aes(x=as.numeric(effect), y= -log10(as.numeric(pvalue)),
                                   colour = sig)) +
  geom_point() +
  labs(x = "Effect", y = expression(-log[10](italic(P)))) +
  # Create the threshold p-value
  geom_hline(yintercept = -log10(1.46e-03), 
             col="darkred", linetype = "dashed") +
  geom_hline(yintercept = -log10(0.05), 
             col="darkgreen", linetype = "dashed") +
  geom_vline(xintercept = 0, col="blue", linetype = "dashed") +
  scale_colour_manual(values = c(`FALSE` = "black", 
                                 `TRUE` = "red")) +
  coord_cartesian(xlim = c(-5, 5)) +
  guides(color = "none", shape = "none") +
  theme_minimal() +
  theme(panel.border = element_rect(colour = "black", 
                                    linewidth = 0.8, fill = NA),
        panel.background = element_rect(fill = "white", 
                                        color = NA),
        plot.background  = element_rect(fill = "white", 
                                        color = NA),
        legend.position = "none") +
  ggrepel::geom_text_repel(data = subset(inexwas_pos_cov, sig), 
                           aes(label = label),
                           size = 4,
                           box.padding = unit(0.35, "lines"),
                           point.padding = unit(0.35, "lines"),
                           max.overlaps  = Inf,
                           segment.color = "black") +
  coord_cartesian(xlim = c(-0.5, 0.5)) +
  labs(x="Effect")

ggsave("post_inexwas_sens.png", post_inexwas_sens, width = 8, height = 9, dpi = 300)
\end{lstlisting}

\subsubsection{MExWAS}
\begin{lstlisting}
X_post_red <- expos(eset_post_ts_cov)

#glmnet
cv_postcovred <- cv.glmnet(as.matrix(X_post_red), Y, family = "gaussian", alpha = 0.5)
enet_postcovred <- glmnet(as.matrix(X_post_red), Y, alpha = 0.5, lambda = cv_postcovred$lambda.min)

#glmnet result 
post_mexwas_sens <- coef(enet_postcovred, s = cv_postcovred$lambda.min) %>% 
  as.matrix() %>% as.data.frame() %>%
  setNames("coef") %>% 
  rownames_to_column("feature") %>%
  filter(feature != "(Intercept)") %>% 
  ggplot(aes(x=feature, y=coef, fill=coef)) +
  geom_col() +
  scale_fill_gradient2(low= "steelblue", mid = "white", 
                       high = "red", midpoint=0) + 
  scale_x_discrete(label = pretty_label) +
  coord_flip() +
  labs(
    x = NULL, y = "Coefficient",
    fill = "Value"
  ) +
  theme_minimal(base_size = 12)

ggsave("post_mexwas_sens.png", post_mexwas_sens, width = 8, height = 9, dpi = 300)
\end{lstlisting}

\subsection{Prenatal}
\subsubsection{Data pre-processing & ExWAS}
\begin{lstlisting}
vars_pre_cov <- desc_noox %>% 
  filter(Status == "prenatal") %>% 
  pull(Name)
vars_pre_cov <- c("bisid", vars_pre_cov)

#Write the variables documents 
pre_vars_cov <- finalvars[ ,vars_pre_cov]
pre_vars_cov[] <- lapply(pre_vars_cov, as.numeric)
write.csv(pre_vars_cov, "exposome_pre_vars_cov_vs.csv", row.names = F)

#Description for postnatal
vars_pre_cov <- vars_pre_cov[vars_pre_cov != "bisid"]
desc_pre_cov <- desc_noox[desc_noox$Name %in% vars_pre_cov,]
write.csv(desc_pre_cov, "description_update_pre_cov.csv")

final_vars_pre_cov_path <- "/Users/victoriasjalimcopy/Documents/BIOX7011/Coding/BIOX7011/Final/exposome_pre_vars_cov_vs.csv"

final_desc_pre_cov_path <- "/Users/victoriasjalimcopy/Documents/BIOX7011/Coding/BIOX7011/Final/description_update_pre_cov.csv"

#using the rexposome
eset_pre_cov <- readExposome(
  exposures = final_vars_pre_cov_path,
  phenotype = final_out_cov_path, 
  description = final_desc_pre_cov_path,
  description.famCol = "Family", 
  description.expCol = "Name",
  exposures.samCol = "bisid",
  phenotype.samCol = "bisid",
  exposures.asFactor = FALSE)
eset_pre_cov

# Transform the continuous data 
eset_pre_t_cov <- trans(eset_pre_cov, fun = expo_best_trans)
#Standardize 
eset_pre_ts_cov <- standardize(eset_pre_t_cov, method = "normal")
eset_pre_ts

# Summary Statistics 
plotFamily(eset_pre_ts_cov, family = "all")

# EWAS 
ex_cum_pre_cov <- exwas(eset_pre_ts_cov, formula = cumwhecgh ~ sex + zscore_bweight, family = "gaussian", warnings = F)
ex_cum_pre_cov
plotExwas(ex_cum_pre_cov)
plotEffect(ex_cum_pre_cov)

#head(extract(inexwas_cum_pre))
exwas_pre_sens <- as.data.frame(ex_cum_pre_cov@comparison) %>% 
   mutate(label = dplyr::recode(rownames(ex_cum_pre_cov@comparison),
                        hresideschild = "Number of children",
                        unhealthy_diet = "Unhealthy Diet",
                        BPA = "BPA",
                        seifa_disad = "SEIFA disadvantages",
                        GlycA_a = "GlycA (pregnancy)",
                        pregalc = "Prenatal Alcohol Intake",
                        healthy_diet = "Healthy Diet"), 
         sig = pvalue < 0.05)

# Volcano plot
exwas_pre_sens <- ggplot(exwas_pre_sens, aes(x=as.numeric(effect), y= -log10(as.numeric(pvalue)),
                                   colour = sig)) +
  geom_point() +
  labs(x = "Effect", y = expression(-log[10](italic(P)))) +
  # Create the threshold p-value
  geom_hline(yintercept = -log10(1.46e-03), 
             col="darkred", linetype = "dashed") +
  geom_hline(yintercept = -log10(0.05), 
             col="darkgreen", linetype = "dashed") +
  geom_vline(xintercept = 0, col="blue", linetype = "dashed") +
  scale_colour_manual(values = c(`FALSE` = "black", 
                                 `TRUE` = "red")) +
  coord_cartesian(xlim = c(-5, 5)) +
  guides(color = "none", shape = "none") +
  theme_minimal() +
  theme(panel.border = element_rect(colour = "black", 
                                    linewidth = 0.8, fill = NA),
        panel.background = element_rect(fill = "white", 
                                        color = NA),
        plot.background  = element_rect(fill = "white", 
                                        color = NA),
        legend.position = "none") +
  ggrepel::geom_text_repel(data = subset(exwas_pre_sens, sig), 
                           aes(label = label),
                           size = 4,
                           box.padding = unit(0.35, "lines"),
                           point.padding = unit(0.35, "lines"),
                           max.overlaps  = Inf,
                           segment.color = "black") +
  coord_cartesian(xlim = c(-0.5, 0.5)) +
  labs(x="Effect")

ggsave("pre_exwas_sens.png", exwas_pre_sens, width = 8, height = 6, dpi = 300)

\end{lstlisting}

\subsubsection{InExWAS}
\begin{lstlisting}
# Inexwas
inexwas_cum_pre_c <- invExWAS(eset_pre_ts_cov, formula = ~sex + zscore_bweight)
#head(extract(inexwas_cum_pre))

inexwas_pre_sens <- as.data.frame(inexwas_cum_pre_c@comparison) %>% 
   mutate(label = dplyr::recode(rownames(inexwas_cum_pre_c@comparison),
                        hresideschild = "Number of children",
                        unhealthy_diet = "Unhealthy Diet",
                        BPA = "BPA",
                        seifa_disad = "SEIFA disadvantages",
                        GlycA_a = "GlycA (pregnancy)",
                        pregalc = "Pregnancy Alcohol Intake",
                        healthy_diet = "Healthy Diet"), 
         sig = pvalue < 0.05)

# Volcano plot
inex_pre_sens <- ggplot(inexwas_pre_sens, aes(x=as.numeric(effect), y= -log10(as.numeric(pvalue)),
                                   colour = sig)) +
  geom_point() +
  labs(x = "Effect", y = expression(-log[10](italic(P)))) +
  # Create the threshold p-value
  geom_hline(yintercept = -log10(1.46e-03), 
             col="darkred", linetype = "dashed") +
  geom_hline(yintercept = -log10(0.05), 
             col="darkgreen", linetype = "dashed") +
  geom_vline(xintercept = 0, col="blue", linetype = "dashed") +
  scale_colour_manual(values = c(`FALSE` = "black", 
                                 `TRUE` = "red")) +
  coord_cartesian(xlim = c(-5, 5)) +
  guides(color = "none", shape = "none") +
  theme_minimal() +
  theme(panel.border = element_rect(colour = "black", 
                                    linewidth = 0.8, fill = NA),
        panel.background = element_rect(fill = "white", 
                                        color = NA),
        plot.background  = element_rect(fill = "white", 
                                        color = NA),
        legend.position = "none") +
  ggrepel::geom_text_repel(data = subset(inexwas_pre_sens, sig), 
                           aes(label = label),
                           size = 4,
                           box.padding = unit(0.35, "lines"),
                           point.padding = unit(0.35, "lines"),
                           max.overlaps  = Inf,
                           segment.color = "black") +
  coord_cartesian(xlim = c(-0.5, 0.5)) +
  labs(x="Effect")

ggsave("pre_inexwas_sens.png", inex_pre_sens, width = 8, height = 10, dpi = 300)

\end{lstlisting}

\subsubsection{MExWAS}
\begin{lstlisting}
X_pre_red <- expos(eset_pre_ts_cov)

#glmnet
cv_precovred <- cv.glmnet(as.matrix(X_pre_red), Y, family = "gaussian", alpha = 0.5)
enet_precovred <- glmnet(as.matrix(X_pre_red), Y, alpha = 0.5, lambda = cv_precovred$lambda.min)

#glmnet result 
pre_mexwas_sens <- coef(enet_precovred, s = cv_precovred$lambda.min) %>% 
  as.matrix() %>% as.data.frame() %>%
  setNames("coef") %>% 
  rownames_to_column("feature") %>%
  filter(feature != "(Intercept)") %>% 
  ggplot(aes(x=feature, y=coef, fill=coef)) +
  geom_col() +
  scale_x_discrete(label = pretty_label) +
  scale_fill_gradient2(low= "steelblue", mid = "white", 
                       high = "red", midpoint=0) + 
  coord_flip() +
  labs(
    x = NULL, y = "Coefficient",
    fill = "Value"
  ) +
  theme_minimal(base_size = 12)

ggsave("pre_mexwas_sens.png", pre_mexwas_sens, width = 8, height = 9, dpi = 300)
\end{lstlisting}

\subsection{MExWAS with covariates}
Combining the list with the covariates
\begin{lstlisting}
evry <- cbind(pre_vars_st, post_vars_st, out_data)
\end{lstlisting}

\subsubsection{Postnatal}
\begin{lstlisting}
post_cov1 <- evry %>% 
  select(sex, zscore_bweight, colnames(post_vars_st))

#glmnet
cv_postcov <- cv.glmnet(as.matrix(post_cov1), evry$cumwhecgh, family = "gaussian", alpha = 0.5)
enet_postcov <- glmnet(as.matrix(post_cov1), evry$cumwhecgh, alpha = 0.5, lambda = cv_postcov$lambda.min)

#glmnet result 
wcov_post <- coef(enet_postcov, s = cv_postcov$lambda.min) %>% 
  as.matrix() %>% as.data.frame() %>%
  setNames("coef") %>% 
  rownames_to_column("feature") %>%
  filter(feature != "(Intercept)") %>% 
  ggplot(aes(x=feature, y=coef, fill=coef)) +
  geom_col() +
  scale_x_discrete(label = pretty_label) +
  scale_fill_gradient2(low= "steelblue", mid = "white", 
                       high = "red", midpoint=0) + 
  coord_flip() +
  labs(
    x = NULL, y = "Coefficient",
    fill = "Value"
  ) +
  theme_minimal(base_size = 12)

ggsave("wcov_post_mexwas.png", wcov_post, width = 8, height = 9, dpi = 300)
\end{lstlisting}

\subsubsection{Prenatal}
\begin{lstlisting}
pre_cov1 <- evry %>% 
  dplyr::select(sex, zscore_bweight, colnames(pre_vars_st))

#glmnet
cv_precov <- cv.glmnet(as.matrix(pre_cov1), evry$cumwhecgh, family = "gaussian", alpha = 0.5)
enet_precov <- glmnet(as.matrix(pre_cov1), evry$cumwhecgh, alpha = 0.5, lambda = cv_precov$lambda.min)

#glmnet result 
wcov_pre <- coef(enet_precov, s = cv_precov$lambda.min) %>% 
  as.matrix() %>% as.data.frame() %>%
  setNames("coef") %>% 
  rownames_to_column("feature") %>%
  filter(feature != "(Intercept)") %>% 
  ggplot(aes(x=feature, y=coef, fill=coef)) +
  geom_col() +
  scale_fill_gradient2(low= "steelblue", mid = "white", 
                       high = "red", midpoint=0) + 
  scale_x_discrete(label = pretty_label) +
  coord_flip() +
  labs(
    x = NULL, y = "Coefficient",
    fill = "Value"
  ) +
  theme_minimal(base_size = 12)

ggsave("wcov_pre_mexwas.png", wcov_pre, width = 8, height = 9, dpi = 300)
\end{lstlisting}

\section{DSA}
Deletion substitution addition algorithm. Based on many papers, especially from HELIX study, they used this method for multivariable exposure analysis. This actually similar to elastic net that created for variable selection for model at the end. 

This one is too long, this is just an example. Would do the rest inside a cluster

We are doing the cross validations first - inside the cluster
This is for running the scripts (.sh) format
\begin{lstlisting}
#!/bin/bash
#SBATCH --nodes=1
#SBATCH --ntasks-per-node=1
#SBATCH --cpus-per-task=32
#SBATCH --mem=48G
#SBATCH --job-name=dsa
#SBATCH --partition=general
#SBATCH --account=a_mcrae
#SBATCH --output=slurm_%j.out
#SBATCH --error=slurm_%j.err
#SBATCH --time=2:00:00

#Load necessary module
module load r/4.4.2

# For the packages
# Set R to use your local library
export R_LIBS_USER="/home/uqvsjali/R/x86_64-pc-linux-gnu-library/4.4"

echo "Using R_LIBS_USER=$R_LIBS_USER"
echo "Installed packages:"
ls "$R_LIBS_USER"
echo "Starting the analysis for postnatal"

#Running the Rscript 
Rscript --vanilla "/home/uqvsjali/dsa/dsa.R"
echo 'R_LIBS_USER="/home/uqvsjali/R/x86_64-pc-linux-gnu-library/4.4"' >  ~/.Renviron
\end{lstlisting}

Cross-validations (using R script)
\begin{lstlisting}
set.seed(177750)

# Library
library(partDSA)
library(caret)
library(ggplot2)
library(tidyverse)
library(parallel) #Make computation time faster 
library(doParallel) 

# Close any existing clusters
try(stopCluster(cl), silent = TRUE)
try(stopImplicitCluster(), silent = TRUE)

# Clear all connections
closeAllConnections()

# setup the parallel computing
cl <- makeCluster(8)
registerDoParallel(cl)

#Reading the data 
#Postnatal 
X_post <- read.csv("/home/uqvsjali/dsa/postnatal_expo.csv")

#Prenatal exposures 
X_pre <- read.csv("/home/uqvsjali/dsa/prenatal_expo.csv")

all <- cbind(X_post, X_pre)

#The outcome 
finalout <- read.csv("/home/uqvsjali/dsa/outcome_update.csv")
Y1 <- finalout$cumwhecgh
Y2 <- finalout$cumwhecgh_cat

#The hyperparameter list 
hyperpar <- expand.grid(
  cut.off.growth = c(10,15,20), #max partitions to be explored
  minsplit = c(20,30), #min number of observation necessary to split
  minbuck = c(2,3,5) #min observations in one partitions 
)

#Function for the tuning
cv_dsa <- function(x, y, formula, params, nfolds = 5) {
  library(parallel)
  library(caret)
  
  n <- nrow(x)
  folds <- createFolds(1:n, k = nfolds, list = TRUE)
  
  # Use parallel processing for cross-validation folds
  cv_errors <- mclapply(1:nfolds, function(i) {
    train_id <- unlist(folds[-i])
    test_id <- folds[[i]]
    
    train_x <- x[train_id, , drop = FALSE]
    test_x <- x[test_id, , drop = FALSE]
    train_y <- y[train_id]
    test_y <- y[test_id]
    
    control_dsa <- DSA.control(
      cut.off.growth = params$cut.off.growth,
      minsplit = params$minsplit,
      minbuck = params$minbuck
    )
    
    # Try fitting and prediction
    result <- tryCatch({
      model_dsa <- partDSA(x = train_x, y = train_y, control = control_dsa)
      pred <- predict(model_dsa, newdata = test_x)
      mean((test_y - pred)^2)
    }, error = function(e) {
      warning(paste("Error in fold", i, ":", e$message))
      NA
    })
    
    return(result)
    
  }, mc.cores = 8)
  
  # Return average error
  return(mean(unlist(cv_errors), na.rm = TRUE))}

#Let's do the tuning
results <- data.frame(hyperpar, cv_error = NA)

for(i in 1:nrow(hyperpar)) {
  cat("Testing parameter set", i, "of", nrow(hyperpar), "\n")
  
  cv_error <- cv_dsa(
    x = all,
    y = Y1,
    formula = Y1 ~ .,
    params = hyperpar[i, ],
    nfolds = 5
  )
  
  results$cv_error[i] <- cv_error
  
  # Print progress
  cat("Parameters:", toString(hyperpar[i, ]), 
      "CV Error:", round(cv_error, 4), "\n")
}

# Find best parameters
best_idx <- which.min(results$cv_error)
best_params <- hyperpar[best_idx, ]
best_error <- results$cv_error[best_idx]
\end{lstlisting}

The best parameters were: minsplit = 30, minbuck = 5, cut.off.growth = 10. We will do a 30 bootstrap samples (I tried just using different seed but it didn't work because they chose the same variables)
\begin{lstlisting}
all <- cbind(X_post, X_pre)
# Create 30 bootstrap samples
B <- 30
# Create a list
runs <- vector("list", B)

for (i in seq_len(B)) {
  set.seed(i)
  # Bootstrapping
  idx <- sample(nrow(all), replace = TRUE)        
  Xb  <- all[idx, , drop = FALSE]
  Yb  <- Y[idx]

    # To know which seeds
  cat("This is seed", i)
  # Run the DSA for each seeds
  fit <- partDSA(
    x = Xb, y = Yb,
    control = DSA.control(minsplit = 30, minbuck = 5, cut.off.growth = 10)
  )
    # For each round put it inside the list 
  runs[[i]] <- list(seed = i, idx = idx, fit = fit)
}

\end{lstlisting}


\begin{lstlisting}
#Find the important variables that selected each runs
var_importances <- lapply(runs, function(x) x$fit$var.importance) 

var_importance_df <- do.call(rbind, lapply(seq_along(var_importances), function(i) {
  v <- var_importances[[i]]
  if (is.null(v)) return(NULL)
  v_df <- as.data.frame(v)
  v_df$variable <- rownames(v_df)
  v_df$i <- i
  v_df
}))

# Grep it from the COG
var_importance_df$sum = rowSums(var_importance_df[ , grep("^COG", names(var_importance_df)) ])

var_select <- var_importance_df %>% filter(sum > 0) %>% 
  select(i, variable, sum) %>% 
  group_by(variable) %>% 
  summarize(count = n()) %>% 
  mutate(percent = count/30*100) %>% 
  filter(count>1)

# Sort it by the most selected
var_select <- var_select[order(var_select$percent, decreasing = T), ]

# Visualisations
ggplot(var_select, aes(y=reorder(variable, percent), x=percent)) +
  geom_col(fill = "steelblue") +
  scale_y_discrete(label = pretty_label) +
  labs(
    x = "Percent selected (%)", y = "Exposure"
  ) +
  theme_minimal(base_size = 12) +
  xlim(0,100)

# Rename it
var_select$variable <- pretty_label(var_select$variable)

write.csv(var_select, "DSA_select.csv")
\end{lstlisting}

\section{Associations}

\subsection{Creating multidatasets}
\begin{lstlisting}
library(devtools)
#devtools::install_github("isglobal-brge/omicRexposome")
#omicRexposome is not in CRAN yet 
library(omicRexposome)
library(MultiDataSet) 
library(SummarizedExperiment)
library(Biobase)

# Data 
final_lipid <- read.csv("/Users/victoriasjalimcopy/Documents/BIOX7011/Coding/BIOX7011/processed_lipid_id.csv", row.names = 1)

#Create the assay 
assay_lipid <- t(as.matrix(final_lipid))

# Pheno data
rownames(out_data1) <- out_data1$id
pheno <- AnnotatedDataFrame(data= out_data1,
          varMetadata = data.frame(
            labelDescription = c("Sample ID",
            "Cumulative wheeze cough
                                 Change",
                                 "Cumulative wheeze cough
                                 Category", 
                                 "Sex (1=Female, 2=Male)",
                                  "Z-score Body Weight"),
              row.names = colnames(out_data1)))

#Feature data - This data consist of the name of our lipid & other info
desc_lipid <- read.csv("/Users/victoriasjalimcopy/Documents/BIOX7011/Coding/BIOX7011/Final/lipid_desc.csv", row.names = 1)

desc_lipid <- desc_lipid[rownames(assay_lipid), , drop = FALSE]

feat_dat <- AnnotatedDataFrame(desc_lipid, 
            varMetadata = data.frame(
              labelDescription = c(
              "Lipid Family Classification",
              "Detailed Lipid Description", "chr","start", "end"),
            row.names = colnames(desc_lipid)))

# Expression set 
lipid_eset <- ExpressionSet(
  assayData = assay_lipid,
  phenoData = pheno,
  featureData = feat_dat,
  annotation = "Lipidomics_MS"
)

# Create the multidatasets
mds <- createMultiDataSet()
mds <- add_eset(mds,lipid_eset,
              dataset.type = "Lipidomics",
              overwrite = T)

# Try combine it with the all data
mds <- add_exp(mds, eset_finalp)
pData(eset_finalp)$id <- sampleNames(eset_finalp)

#There is a mismatch ID problem - debugging 
# can't do further analysis because of this 
common_id <- commonIds(mds) 
length(common_id) #same length 

#Reorder the ID variable  
# Sort numerically
numeric_order <- common_id[order(as.numeric(common_id))]
\end{lstlisting}

\subsection{Associations}
\begin{lstlisting}
# Associations 
test_assoc <- association(mds,~sex + zscore_bweight,
                           expset = "exposures",
                           omicset = "Lipidomics")

hit <- tableHits(test_assoc, th=0.001)
lab <- tableLambda(test_assoc)
result_ass <- merge(hit, lab, by="exposure") %>% 
  arrange(result_ass)

#RUN THIS IF HAVE TIME - Adding another SVA analysis (more robust)
test_sva <- association(mds,~sex + zscore_bweight,
                          expset = "exposures",
                        omicset = "Lipidomics",
                        sva = "slow")
test_sva@results
# Calculate how many have hits 
hit_sva <- tableHits(test_sva, th=0.001)
lab_sva <- tableLambda(test_sva)
sva <- merge(hit_sva, lab_sva, by="exposure")

sva$exposure <- pretty_label(sva$exposure)
sva$variable <- sva$exposure 
sva <- merge(fam, sva, by="variable")

write.csv(sva, "associations_result.csv")
\end{lstlisting}

\subsubsection{Saving the data}
Need to find the positions first to be call out from the analysis
\begin{lstlisting}
# Atregs
which(names(test_sva) == "atreg12m") #8 
atregs_ass <- as.data.frame(getAssociation(test_sva, rid = 8))
write.csv(atregs_ass, "atregs_ass.csv")


#glyca
which(names(test_sva) == "GlycA12m") #43
glyca_ass <- as.data.frame(getAssociation(test_sva, rid = 43))
write.csv(glyca_ass, "glyca_ass.csv")

#numchild
which(names(test_sva) == "hresideschild") #49
numchild_ass <- as.data.frame(getAssociation(test_sva, rid = 49))
write.csv(numchild_ass, "numchild_ass.csv")

#pss 
which(names(test_sva) == "pss_m12") #78
pss_ass <- as.data.frame(getAssociation(test_sva, rid = 78))
write.csv(pss_ass, "pss_ass.csv")
\end{lstlisting}

\subsubsection{Plotting the graph}
\begin{lstlisting}
pval <- -log10(0.001)
#plot the protein 
gridExtra::grid.arrange(
    plotAssociation(test_sva, rid="GlycA12m", type="protein") + 
    ggplot2::ggtitle("GlycA (12m)") +
    ggplot2::geom_hline(yintercept = pval, color = "LightPink"),
    plotAssociation(test_sva, rid="atreg12m", type="protein") + 
      ggplot2::ggtitle("aTregs (12m)") +
      ggplot2::geom_hline(yintercept = pval, color = "LightPink"),
    plotAssociation(test_sva, rid="hresideschild", type="protein") + 
      ggplot2::ggtitle("Number of Children") +
      ggplot2::geom_hline(yintercept = pval, color = "LightPink"),
    plotAssociation(test_sva, rid="pss_m12", type="protein") + 
      ggplot2::ggtitle("PSS Score(12m)") +
      ggplot2::geom_hline(yintercept = pval, color = "LightPink"),
    ncol=3)

# Plot the volcano 
vol_ass <- gridExtra::grid.arrange(
    plotAssociation(test_sva, rid="GlycA12m", type="volcano", tPV=pval)+
      ggplot2::ggtitle("GlycA \n at 12m") + 
      ggplot2::theme(plot.title = element_text(hjust = 0.5)) +
      ggplot2::ylim(0,25),
    plotAssociation(test_sva, rid="atreg12m", type="volcano", tPV=pval)+
      ggplot2::ggtitle("aTregs \n at 12m") + 
      ggplot2::theme(plot.title = element_text(hjust = 0.5)) +
      ggplot2::ylim(0, 25),
    plotAssociation(test_sva, rid="hresideschild", type="volcano", tPV=pval) + 
      ggplot2::ggtitle("Number of \n Children") +
      ggplot2::theme(plot.title = element_text(hjust = 0.5)) +
      ggplot2::ylim(0, 25),
     plotAssociation(test_sva, rid="pss_m12", type="volcano", tPV=pval) + ggplot2::ggtitle("PSS Score \n at 12m") + 
      ggplot2::theme(plot.title = element_text(hjust = 0.5)) +
      ggplot2::ylim(0, 25) +
      ggplot2::geom_hline(yintercept = -log10(0.001), 
                          colour ="grey", linetype ="twodash"), ncol=4)

ggsave(filename = "vol_ass.png", vol_ass, width = 8, height = 9, dpi = 300)

# Plot qq 
gridExtra::grid.arrange(
    plotAssociation(test_sva, rid="GlycA12m", type="qq")+
      ggplot2::ggtitle("GlycA \n at 12m") + 
      ggplot2::theme(plot.title = element_text(hjust = 0.5)),
    plotAssociation(test_sva, rid="atreg12m", type="qq")+
      ggplot2::ggtitle("aTregs \n at 12m") + 
      ggplot2::theme(plot.title = element_text(hjust = 0.5)),
    plotAssociation(test_sva, rid="hresideschild", type="qq") + 
      ggplot2::ggtitle("Number of \n Children") +
      ggplot2::theme(plot.title = element_text(hjust = 0.5)),
     plotAssociation(test_sva, rid="pss_m12", type="qq") + ggplot2::ggtitle("PSS Score \n at 12m") + 
      ggplot2::theme(plot.title = element_text(hjust = 0.5)),
    plotAssociation(test_sva, rid="healthy_diet", type="qq") + 
      ggplot2::ggtitle("Healthy Diet") +
      ggplot2::theme(plot.title = element_text(hjust = 0.5)), ncol=5)
\end{lstlisting}

\section{Mediation}
\begin{lstlisting}
library(mediation)

# Write functions for automatic mediations 
quick_mediation <- function(data, x, y, mediators, covars = NULL) {
  
  med_names <- colnames(mediators)
 
  out <- lapply(med_names, function(med) {
     cat("\n=== Testing mediator:", med, "===\n")
    tryCatch({
      if (is.null(covars)) {
        # Build formulas as strings first
        f_m_str <- paste(med, "~", x)
        f_y_str <- paste(y, "~", med, "+", x)
      } else {
        rhs_vars <- paste(c(x, covars), collapse = " + ")
        f_m_str <- paste(med, "~", rhs_vars)
        f_y_str <- paste(y, "~", med, "+", rhs_vars)
      }
      
      # Convert to formulas and fit models
      f_m <- as.formula(f_m_str)
      f_y <- as.formula(f_y_str)
      
      mod.m <- lm(f_m, data = data)
      mod.y <- lm(f_y, data = data)
      
      mod.m$call$formula <- f_m
      mod.y$call$formula <- f_y
      
      # Run mediation - this is the key fix
      fit <- mediation::mediate(
        model.m = mod.m,
        model.y = mod.y,
        treat = x,
        mediator = med,
        sims = 500, 
        boot = T
      )
      cat("Mediation completed for", med, "\n")
      
      # Extract results
      data.frame(
        mediator = med,
        ACME = fit$d0,
        ACME_CI_L = fit$d0.ci[1],
        ACME_CI_U = fit$d0.ci[2],
        ACME_p = fit$d0.p,
        ADE = fit$z0,
        ADE_CI_L = fit$z0.ci[1],
        ADE_CI_U = fit$z0.ci[2],
        ADE_p = fit$z0.p,
        TotalEffect = fit$tau.coef,
        Total_CI_L = fit$tau.ci[1],
        Total_CI_U = fit$tau.ci[2],
        Total_p = fit$tau.p,
        PropMediated = fit$n0,
        PropMed_CI_L = fit$n0.ci[1],
        PropMed_CI_U = fit$n0.ci[2],
        PropMed_p = fit$n0.p,
        stringsAsFactors = FALSE,
      )
      
    }, error = function(e) {
      # Return empty row if mediation fails
      cat("Error with mediator", med, ":", e$message, "\n")
      return(NULL)
    })
  })
  
  # Remove NULL results and bind rows
  out <- out[!sapply(out, is.null)]
  if(length(out) > 0) {
    do.call(rbind, out)
  } else {
    cat("No successful mediations\n")
    return(data.frame())
  }
}

# GlycA
medglyc <- quick_mediation(evry,"GlycA12m", "cumwhecgh", final_lipid,covars = c("sex","zscore_bweight"))
medglyc <- medglyc %>% 
  mutate(p.adj = p.adjust(ADE_p, method = "fdr"))
write.csv(medglyc, "Mediations_glyca.csv", row.names = F)


# atregs
medatregs <- quick_mediation(evry, "atreg12m", "cumwhecgh",
                             final_lipid, covars = c("sex","zscore_bweight"))
medatregs <- medatregs %>% 
  mutate(p.adj = p.adjust(ADE_p, method = "fdr"))
write.csv(medatregs, "Mediation_atregs.csv", row.names = F)

#N children
med_hresides <- quick_mediation(evry,"hresideschild", "cumwhecgh", final_lipid,covars = c("sex","zscore_bweight")) 
#Adjusting p-value 
med_hresides <- med_hresides %>% 
  mutate(p.adj = p.adjust(ADE_p, method = "fdr"))
write.csv(med_hresides, "Mediation_nchildren.csv", row.names = F)

#Diet 
med_diet <- quick_mediation(evry,"healthy_diet", "cumwhecgh", final_lipid,covars = c("sex","zscore_bweight"))
med_diet <- med_diet %>% 
  mutate(p.adj = p.adjust(ADE_p, method = "fdrddad"))
write.csv(med_diet, "Mediation_healthydiet.csv", row.names = F)

# PSS  
med_pss <- quick_mediation(evry,"pss_m12", "cumwhecgh", final_lipid,covars = c("sex","zscore_bweight"))
med_pss <- med_pss %>% 
  mutate(p.adj = p.adjust(ADE_p, method = "fdr"))
write.csv(med_pss, "Mediation_pss.csv", row.names = F)
\end{lstlisting}

\section{Partial correlations}
\begin{lstlisting}
# Select the significant data 
mds_ord@assayData$exposures$exp
out_id <- mds_ord@phenoData$exposures$main@data %>% 
  select(id, cumwhecgh)

all_id <- as.data.frame(t(mds_ord@assayData$exposures$exp))
all_id <- rownames_to_column(all_id, var="id")

all_data_phen <- merge(all_id, out_id, by = "id")
sel_sig <- all_data_phen %>% 
  select(hresideschild, pregalc, healthy_diet, GlycA12m, fumany, msmoke_m12, pss_m12, cumwhecgh) 

colnames(sel_sig) <- pretty_label(colnames(sel_sig))

# Visualisations
# Heatmap
library(corrplot)

estimate_cor <- all_pcor$est
p_cor <- all_pcor$pval

png("corrplot_pcor.png", width = 9, height = 9, units = "in", res = 300)

# Hide non-significant cells (FDR ≥ 0.05)
cor_all <- corrplot(estimate_cor,
         method = "color",
         type = "full",
         tl.col = "black",
         tl.srt = 45,
         p.mat = p_cor,
         sig.level = 0.05,
         insig = "blank",
         addgrid.col = NA,
         addCoef.col = "black",
         number.cex = 0.7,
         col = colorRampPalette(c("steelblue", "white", "firebrick"))(200))

dev.off()

corrplot(cor_pht)

# The network
library(ggraph)
library(tidygraph)

# Choose edges (find the significant one) 
mask <- (p_cor < 0.05)
edges <- which(mask & upper.tri(estimate_cor), arr.ind = TRUE)

edgedf <- data.frame(
  from = rownames(estimate_cor)[edges[,1]],
  to   = colnames(estimate_cor)[edges[,2]],
  r    = estimate_cor[edges]
)

# Build graph
g <- graph_from_data_frame(edgedf, directed = FALSE)

g_tbl <- as_tbl_graph(g)

ggraph(g_tbl, layout = "fr") +
  geom_edge_link(aes(width = abs(r), color = r > 0), alpha = 0.6) +
  geom_node_point(size = 5, color = "grey80") +
  geom_node_text(aes(label = name), repel = TRUE, size = 3,
                 point.padding = unit(0.25, "lines"),
                 box.padding   = unit(0.5, "lines"),
                 segment.size = 0) +
  scale_edge_width(range = c(0.4, 3)) +
  scale_edge_color_manual(values = c("steelblue", "firebrick"),
                          labels = c("Negative", "Positive")) +
  theme_void() +
  theme(legend.position = "bottom")
\end{lstlisting}

\end{document}


